\PassOptionsToPackage{unicode=true}{hyperref} % options for packages loaded elsewhere
\PassOptionsToPackage{hyphens}{url}
%
\documentclass[12,]{article}
\usepackage{lmodern}
\usepackage{amssymb,amsmath}
\usepackage{ifxetex,ifluatex}
\usepackage{fixltx2e} % provides \textsubscript
\ifnum 0\ifxetex 1\fi\ifluatex 1\fi=0 % if pdftex
  \usepackage[T1]{fontenc}
  \usepackage[utf8]{inputenc}
  \usepackage{textcomp} % provides euro and other symbols
\else % if luatex or xelatex
  \usepackage{unicode-math}
  \defaultfontfeatures{Ligatures=TeX,Scale=MatchLowercase}
\fi
% use upquote if available, for straight quotes in verbatim environments
\IfFileExists{upquote.sty}{\usepackage{upquote}}{}
% use microtype if available
\IfFileExists{microtype.sty}{%
\usepackage[]{microtype}
\UseMicrotypeSet[protrusion]{basicmath} % disable protrusion for tt fonts
}{}
\IfFileExists{parskip.sty}{%
\usepackage{parskip}
}{% else
\setlength{\parindent}{0pt}
\setlength{\parskip}{6pt plus 2pt minus 1pt}
}
\usepackage{hyperref}
\hypersetup{
            pdftitle={Population-weighted exposure to air pollution and COVID-19 infection in Germany},
            pdfauthor={Guowen Huang\^{}\{1,2,*\}, Patrick E Brown\^{}\{1,2\}},
            pdfborder={0 0 0},
            breaklinks=true}
\urlstyle{same}  % don't use monospace font for urls
\usepackage{graphicx,grffile}
\makeatletter
\def\maxwidth{\ifdim\Gin@nat@width>\linewidth\linewidth\else\Gin@nat@width\fi}
\def\maxheight{\ifdim\Gin@nat@height>\textheight\textheight\else\Gin@nat@height\fi}
\makeatother
% Scale images if necessary, so that they will not overflow the page
% margins by default, and it is still possible to overwrite the defaults
% using explicit options in \includegraphics[width, height, ...]{}
\setkeys{Gin}{width=\maxwidth,height=\maxheight,keepaspectratio}
\setlength{\emergencystretch}{3em}  % prevent overfull lines
\providecommand{\tightlist}{%
  \setlength{\itemsep}{0pt}\setlength{\parskip}{0pt}}
\setcounter{secnumdepth}{5}
% Redefines (sub)paragraphs to behave more like sections
\ifx\paragraph\undefined\else
\let\oldparagraph\paragraph
\renewcommand{\paragraph}[1]{\oldparagraph{#1}\mbox{}}
\fi
\ifx\subparagraph\undefined\else
\let\oldsubparagraph\subparagraph
\renewcommand{\subparagraph}[1]{\oldsubparagraph{#1}\mbox{}}
\fi

% set default figure placement to htbp
\makeatletter
\def\fps@figure{htbp}
\makeatother

\usepackage{subcaption}
\newcommand{\subfloat}[2][need a sub-caption]{ \subcaptionbox{#1}{#2} }
\usepackage{xcolor}
\usepackage{setspace}\doublespacing
\usepackage{pdflscape}
\usepackage{float}
\usepackage{booktabs}
\newcommand{\tp}{^\text{T}}
\newcommand{\dens}{\boldsymbol{\pi}}
\usepackage[a4paper,width=15cm, lines=26, left=3cm ]{geometry}

\usepackage[style=authoryear,backend=biber,maxbibnames = 20,maxcitenames = 2,doi = false,isbn = false,giveninits = true,uniquelist = false]{biblatex}
\addbibresource{thebib.bib}

\title{Population-weighted exposure to air pollution and COVID-19 infection in
Germany}
\author{Guowen Huang\(^{1,2,*}\), Patrick E Brown\(^{1,2}\)}
\date{}

\begin{document}
\maketitle

\hypertarget{abstract}{%
\subsection*{Abstract}\label{abstract}}
\addcontentsline{toc}{subsection}{Abstract}

It is well known that COVID-19, caused by the severe acute respiratory
syndrome coronavirus 2 (SARS-CoV-2), is to spread mainly from person to
person, mainly through respiratory droplets produced when an infected
person coughs or sneezes. Therefore, many countries have enforced social
distancing to stop the spread of COVID-19. Within countries, although
the measures taken by governments are similar, the incidence rate varies
among areas (e.g., counties, cities). One potential explanation is that
people in some areas are more vulnerable to the coronavirus disease
because of their worsened health conditions caused by long-term exposure
to poor air quality. In this study, we investigate whether long-term
exposure to air pollution increases the risk of COVID-19 infection in
Germany. The results show that nitrogen dioxide (NO\(_2\)) is
significantly associated with COVID-19 incidence rate, with 1
\(\mu gm^{-3}\) increase of long-term exposure to NO\(_2\), the COVID-19
incidence rate is likely to increase 5.58\% (95\% credible interval
{[}CI{]}: 3.35\%, 7.86\%). This result is consistent across various
health models. The analyses can be reproduced and updated routinely
using public data sources and shared R code.

\hypertarget{keywords}{%
\subsection*{Keywords}\label{keywords}}
\addcontentsline{toc}{subsection}{Keywords}

COVID-19, air pollution, health impacts, Kriging, INLA

\hypertarget{author-information}{%
\subsection*{Author information}\label{author-information}}
\addcontentsline{toc}{subsection}{Author information}

\(^1\) Dept. of Statistical Sciences, University of Toronto, Toronto,
ON, Canada\\
\(^2\) Centre for Global Health Research, St Michael's Hospital,
Toronto, ON, Canada\\
\(^*\) Corresponding author: E-mail: hgw0610209@gmail.com

\hypertarget{introduction}{%
\section{Introduction}\label{introduction}}

COVID-19, caused by the severe acute respiratory syndrome coronavirus 2
(SARS-CoV-2), is currently widespread and much more dangerous than
seasonal flu. It is more infectious than seasonal flu and has a higher
death rate. Up to 17th September, 2020, it has led to worldwide over
30.3 million cases and 950 thousand deaths. In Germany, the total
confirmed cases up to 17th September, 2020 have risen to 269 thousand,
with deaths being more than 9.4 thousand. A recent study by
\textcite{Wu2020} investigated the impact of long-term average exposure
to fine particulate matter (PM\(_{2.5}\)) on the risk of COVID-19 deaths
in the United States and found that an increase of only 1
\(\mu g/m^{-3}\) in PM\(_{2.5}\) was associated with a 8\% (95\%
confidence interval, 2\%, 15\%) increase in the COVID-19 death rate.
\textcite{Ogen2020} reported that most of COVID-19 fatality cases
occurred in those regions with the highest NO\(_2\) concentrations while
studying 66 administrative regions in Italy, Spain, France and Germany.
These results suggest that high levels of air pollution may be an
important contributor to COVID-19 infections or deaths.

\hypertarget{literature-review-on-epidemiology}{%
\subsection{Literature review on
epidemiology}\label{literature-review-on-epidemiology}}

The existing body of research on the impacts of air pollution on human
health, has linked PM\(_{2.5}\) and NO\(_2\) exposure to health damage,
particularly respiratory and lung diseases, which could make people more
vulnerable to contracting COVID-19. The main source of NO\(_2\)
resulting from human activities is the combustion of fossil fuels (coal,
gas and oil), especially fuel used in cars. Exposure to high levels of
NO\(_2\) can cause inflammation of the airways. Long-term exposure may
affect lung function and respiratory symptoms. For example, the research
results from \textcite{Bowatte2017} indicate that long-term exposure to
NO\(_2\) was associated with increased risk of respiratory diseases,
while \textcite{Lee2009} show that long-term exposure to NO\(_{2}\) was
significantly associated with respiratory hospital admissions in
Edinburgh and Glasgow, UK. Similarly, \textcite{Schikowski2005} suggests
that long-term exposure to air pollution from NO\(_2\) and living near a
major road might increase the risk of developing chronic obstructive
pulmonary disease (COPD) and can have a detrimental effect on lung
function.

On the other hand, particulate matter (both PM\(_{10}\) and
PM\(_{2.5}\)) is made up of a wide range of materials and arises from
both human-made (such as stationary fuel combustion and transport) and
natural sources (such as sea spray and Saharan sand dust).
Concentrations of particulate matter comprises primary particles emitted
directly into the atmosphere from combustion sources and secondary
particles formed by chemical reactions in the air. Exposure to
particulate matter is associated with respiratory and cardiovascular
illness and mortality as well as other adverse health effects. Since the
particulate matter can be inhaled into the thoracic region of the
respiratory tract, there is a plausible reason the relationship could be
causal. Examples include \textcite{Lee2009} and \textcite{Lee2012b},
where the authors found that long-term exposure to PM\(_{10}\) was
significantly associated with respiratory hospital admissions. Recent
reviews by \textcite{COMEAP2010} have suggested exposure to PM\(_{2.5}\)
had a stronger association with the observed adverse health effects
because they can travel deeper into lungs.

\hypertarget{literature-review-on-statistical-models}{%
\subsection{Literature review on statistical
models}\label{literature-review-on-statistical-models}}

A spatial ecological design can be used to estimate the impacts of air
pollution on health by comparing geographical contrasts in air pollution
and infection risk across \(K\) contiguous small areas
\autocites{Huang2018}{Napier2018}{Rushworth2014}. In such studies, the
disease data are counts of disease cases occurring in each areal unit
while the spatially representative pollution concentrations in each
areal unit are typically estimated by applying Kriging, a spatial model
\autocite{Diggle2007}, to data from a sparse monitoring network, or by
computing averages over modelled concentrations (grid level) from an
atmospheric dispersion model
\autocites{Wu2020}{Maheswaran2006}{Lee2009}{Warren2012a}, or by
combining both to obtain a better prediction
\autocites{Huang2018}{VinikoorImler2014}{Sacks2014}. The downside of
these studies is that the inference is a population level association
rather than an individual-level causal relationship, and wrongly
assuming the two are the same is known as ecological bias
\autocites{Arbia1988}{Wakefield2001}. Such bias is due in part to
within-population variation in pollution exposures and disease
incidence, because one does not know whether, within a population, it is
the same individuals that exhibit disease and have the highest air
pollution exposures \autocite{Lee2020}. The simulation study from
\textcite{Lee2020} also suggests that the estimates of the aggregated
model from individual levels almost always exhibit less variation than
those from the ecological model.

Another challenge in air pollution health effect studies is how to allow
for the uncertainty in the estimated pollution concentrations when
estimating their health effects \autocites{Huang2018}{Blair2007}.
Specifically, the areal level pollution predictions produced from the
pollution data are uncertain as they are only estimates of the true
spatially-varying concentrations. The disadvantage of using a point
estimate is that one may overstate the certainty about the connection
between the outcome and the covariate. A number of approaches have been
proposed to incorporate pollution uncertainties and measurement errors
into the health model
\autocites[e.g.,][]{Huang2018}{Lee2017}{Blangiardo2016}{Gryparis2009}.

In this study, we investigate whether long-term average exposure to air
pollution increases the risk of COVID-19 infection in Germany using a
spatial ecological design. Specifically, in order to reduce the
potential ecological bias, we better estimate the true areal pollution
concentrations by first using Kriging to pollution monitoring data to
obtain predictions on fine grids where population density data are
available, then estimate the areal pollution concentrations by taking
spatially population-weighted average of the gridded predictions lying
within a specific county. This will likely enhance the estimation of
people's real exposure for those counties where they generally live at
rural areas while their urban pollution are much worse compared to the
rural areas. Given that the study from \textcite{Lee2017} showed that
treating the posterior predictive pollution distribution as a prior in
the disease model has produced similar results to ignoring the
uncertainty except for PM\(_{10}\), and \textcite{Blangiardo2016} also
found that incorporating uncertainty in pollution by making multiple
sets of estimated exposure and then fitting the disease model separately
for each set before combining the estimated health effects, did not
change the substantive conclusions, we do not address exposure
uncertainty in this study. Instead, we incorporate the reliability of
gridded pollution predictions while aggregating them spatially, with
details can be found in Section \ref{sec:weightedexposure}.

The remainder of this paper is organized as follows. The data and its
exploratory analysis are presented in Section \ref{sec:studyRegion},
while the statistical methodology is outlined in Section
\ref{sec:method}. The results of the study are reported in Section
\ref{sec:results}, and the key conclusions are presented in Section
\ref{sec:conclusion}.

\hypertarget{sec:studyRegion}{%
\section{Study region}\label{sec:studyRegion}}

The study region is Germany which has a population of around 83 million
people and \(K=401\) counties (administrative districts), among which
294 are rural and 107 are urban. A map of these counties is shown in
Figure \ref{fig:maps}, showing boundaries obtained from Germany's
Federal Agency for Cartography and Geodesy \autocite{BKG2020}.

\hypertarget{data-description}{%
\subsection{Data description}\label{data-description}}

The data set used in this study include COVID-19 cases, pollution
concentrations, temperature and population data. The accumulated
COVID-19 cases used in this study are collected up to 2020-09-13 at the
county-level. Both pollution and temperature data are the most recent
available few years (2016-2018) average concentrations (representing
long-term exposure) from monitoring sites, which are converted to
county-level by applying the spatial modelling and prediction method
described in Section \ref{sec:method} to obtain the spatially
population-weighted representative concentrations for each county. The
pollutants considered in this study include common pollutants:
PM\(_{2.5}\) and PM\(_{10}\), NO\(_{2}\), SO\(_2\); and also four
poisonous pollutants: benzene, arsenic in PM\(_{10}\), cadmium in
PM\(_{10}\) and nickel in PM\(_{10}\). These pollutants could have
potential harmful health effects, such as damage to the lungs and nasal
cavity, reducing lung function, causing chronic bronchitis and cancers
of the bladder and lungs
\autocites{Yu2002}{Smith2010}{Jarup1998}{Das2008}.

The population data contain fine gridded population density data, the
population by sex and age on the federal state level and also the county
level population data. The fine gridded population density data are used
for calculating population-weighted county level exposure (see Section
\ref{sec:method} for details), while the latter two are used to
calculate the expected number of cases in each county. Specifically, we
denote Y\(_k\) as the reported numbers of COVID-19 cases for county
\(k\), and calculate the expected number of cases in each county by
\(E_k=\frac{P_k}{P_{s(k)}} \sum_j r_j P_{s(k),j}\), where \(P_k\) is the
population in county \(k\), \(r_j\) is the national incidence rate in
sex-age group \(j\), and \(P_{s(k)}= \sum_jP_{s(k),j}\) denotes the
population of the state which contains county \(k\). The latter part in
the equation \(\sum_j r_j P_{s(k),j}\) is the expected number of cases
in state \(s(k)\). Then we use standardized incidence ratio (SIR) given
by SIR\(_{k}=Y_{k}/E_{k}\), to measure the risk of disease, and an SIR
of 1.1 indicates a 10\(\%\) increased risk of disease compared to that
expected. A spatial map of the natural logarithm of SIR for COVID-19
(the scale will be modelled on) as of 2020-09-13 can be seen in Figure
\ref{fig:maps3}, showing a wide variation in SIRs across the counties in
Germany and the majority of the high‐risk counties are at the east and
south part of Germany.

\hypertarget{data-sources}{%
\subsection{Data sources}\label{data-sources}}

The COVID-19 cases by county, and the population by sex and age on the
federal state level in Germany are publicly available on Kaggle
\autocite{HeadsorTails2020}, where the COVID-19 cases and deaths will be
updated daily, with the earliest recorded cases are from 2020-01-24.
More details on the original sources of COVID-19 and state level
population data can be found in \textcite{HeadsorTails2020}. The county
level population data are freely obtained from the City Population
\autocite{Citypopulation2019}. Both population data sets reflect the
(most recent available) estimates on 2018-12-31. The fine gridded
population density data are freely available on DIVA-GIS
\autocite{DIVAGIS2020}, which is shown in Figure \ref{fig:maps2}.

Pollution data are available from the Air Quality e-Reporting provided
by European Environment Agency \autocite{EEA2020}, where the monitoring
stations are shown in Figure \ref{fig:maps1} which tend to be dense
where the population density is high (see Figure \ref{fig:maps2}). The
monitoring temperature data can be freely downloaded from European
Climate Assessment \& Dataset \autocite{ECAD2020}.

\begin{figure}[H]
\subfloat[Pollution stations\label{fig:maps1}]{\includegraphics[width=0.5\textwidth]{figure/maps-1} }\subfloat[Population density\label{fig:maps2}]{\includegraphics[width=0.5\textwidth]{figure/maps-2} }\subfloat[log(SIR) in counties\label{fig:maps3}]{\includegraphics[width=0.5\textwidth]{figure/maps-3} }\subfloat[NO$_2$ in counties\label{fig:maps4}]{\includegraphics[width=0.5\textwidth]{figure/maps-4} }\caption[Pollution stations, population density, log COVID-19 SIR and population-weighted NO$_2$ ($\mu g m^{-3}$) in Germany]{Pollution stations, population density, log COVID-19 SIR and population-weighted NO$_2$ ($\mu g m^{-3}$) in Germany.}\label{fig:maps}
\end{figure}

\hypertarget{sec:method}{%
\section{Method}\label{sec:method}}

Based on the observed and expected counts of disease cases occurring in
each areal unit, we calculate SIR\(_{k}=Y_{k}/E_{k}\), to measure the
risk of disease. SIR\(_{k}>1\) represents areas with elevated levels of
disease risk, while SIR\(_{k}<1\) corresponds to comparatively healthy
areas. The elevated risks are likely to happen by chance if \(E_{k}\) is
small, which can occur if the disease in question is rare and/or the
population at risk is small \autocite{Lee2011}. To overcome this
problem, the Poisson log-linear spatial models are typically used for
the analysis \autocites{Elliott2000}{Banerjee2004}{Lawson2008}, where
the linear predictor includes pollutant concentrations and potential
confounders. These known covariates are augmented by a set of random
effects to capture the residual spatial autocorrelation after the
covariate effects have been accounted for. The random effects borrow
strength from values in neighbouring areas, which reduces the likelihood
of excesses in risk occurring by chance.

These random effects are commonly modelled by the class of conditional
autoregressive (CAR) prior distributions, which are a type of Markov
random field model \autocite{Rue2005}. The spatial correlation between
the random effects is determined by a binary \(K × K\) neighbourhood
matrix \textbf{W}. Based on this neighbourhood matrix, the most common
models for the random effects include intrinsic autoregressive
\autocite{Besag1991}, convolution model \autocite{Besag1991}, as well as
those proposed by \textcite{Cressie1993} and \textcite{Leroux1999}.
These CAR models differ by holding different assumptions about how the
random effects depend on each other across space.

\hypertarget{pollution-model}{%
\subsection{Pollution model}\label{pollution-model}}

For simplicity, in this study we use a univariate model for each
pollutant, since the number of monitoring stations is 709 which is a
large sample size and produces predictions with modest standard errors.
We treat the underlying pollution levels in Germany as a spatial
Gaussian process \(\{S(\symbf{s}):\symbf{s}\in \mathbb{R}^{2}\}\) with
mean \(\mu\), variance \(\sigma^2=\mbox{Var}\{S(\symbf{s})\}\) and
correlation function
\(\rho(u)= \mbox{Corr}\{S(\symbf{s}), S(\symbf{s}\prime)\}= \mbox{exp}(-u/\eta)\),
where \(u=\lVert \symbf{s}-\symbf{s}\prime\rVert\) denotes the Euclidean
distance between \(\symbf{s}\) and \(\symbf{s}\prime\). Denote the
observed pollution data as
\(\symbf{Z}=\{Z(\symbf{s}); \symbf{s}= \symbf{s}_1,\dots,\symbf{s}_n\}\),
and write
\(\symbf{S}=\{S(\symbf{s}); \symbf{s}= \symbf{s}_1,\dots,\symbf{s}_n\}\)
for the unobserved values of the signal at the sampling locations
\(\symbf{s}_1,\dots,\symbf{s}_n\), the pollution model is assumed as
\begin{equation}
\symbf{Z} = \symbf{S} + \symbf{\epsilon},
\label{eq:model}
\end{equation} where
\(\symbf{\epsilon} (\symbf{s}) \sim \mbox{N} (\symbf{0}, \tau^2\symbf{I})\)
is uncorrelated with \(\symbf{S}\), and \(\symbf{I}\) is the identity
matrix of order \(n\). \(\symbf{S}\) is multivariate Gaussian with mean
vector \(\mu \symbf{1}\), where \(\symbf{1}\) denotes a vector each of
whose elements is 1, and variance matrix \(\tau^2 R\), where \(R\) is
the \(n\) by \(n\) matrix with elements
\(r_{ij}=\rho(\lVert \symbf{s}_i-\symbf{s}_j\prime\rVert)\). Similarly,
\(Z\) is multivariate Gaussian \begin{eqnarray}
\symbf{Z} &\sim& \mbox{N} (\mu\symbf{1}, \sigma^2 V)\\\nonumber
V &=&  R+\nu^2\symbf{I},
\label{eq:var}
\end{eqnarray} where \(\nu^2=\tau^2/\sigma^2\) is the noise-to-signal
variance ratio.

The log-likelihood function of (\ref{eq:model}) is \begin{eqnarray}
L(\mu, \tau^2,\sigma^2,\eta) = -0.5\{n\log(2\pi)+\log\{\mid \sigma^2 R(\eta)+\tau^2\symbf{I} \mid\}+(\symbf{Z}- \mu\symbf{1})\prime ( \sigma^2 R(\eta)+\tau^2\symbf{I})^{-1} (\symbf{Z} - \mu\symbf{1})\}
\label{eq:likelihood}
\end{eqnarray}

Given \(V\), the maximum likelihood estimate (mle) of \(\mu\) and
\(\sigma^2\) is given by, \begin{eqnarray}
\hat{\mu} (V) &=& (\symbf{1}\prime V^{-1}\symbf{1})^{-1}\symbf{1}\prime V^{-1}\symbf{Z}\\\nonumber
\hat{\sigma}^2(V) &=& n^{-1}(\symbf{Z}- \hat{\mu}\symbf{1})\prime V^{-1} (\symbf{Z} - \hat{\mu}\symbf{1}).
\label{eq:gle}
\end{eqnarray}

By substituting \(\hat{\mu} (V)\), and \(\hat{\sigma}^2 (V)\) into the
log-likelihood function, we have, \begin{eqnarray}
L_0(\nu^2,\eta) = -0.5\{n\log(2\pi)+n\log \hat{\sigma}^2(V) + \log{\mid V\mid}+n\},
\label{eq:likelihood0}
\end{eqnarray} which can be optimized numerically with respect to
\(\eta\) and \(\nu\), followed by back substitution to obtain
\(\hat{\sigma}^2\) and \(\hat{\mu}\). This is achieved by function
\textbf{likfit()} in \textbf{geoR} package by providing initial values
for the covariance parameters \autocite{Diggle2007}.

\hypertarget{sec:weightedexposure}{%
\subsection{Population-weighted exposure}\label{sec:weightedexposure}}

The areal pollution exposure is estimated by aggregating the gridded
predictions weighted by population density and also the precision of
prediction. For a new location \(\symbf{s}_0\), the Kriging formula of
\(S(\symbf{s}_0)\) \autocite{Diggle2007} is used to obtain its
prediction by plugging-in the resulting estimates
\(\hat{\mu}, \sigma^2, \hat{\tau}^2, \hat{\eta}\), which is
\begin{equation}
\hat{S}(\symbf{s}_0) = \hat{\mu}+ \symbf{C}\prime_{\symbf{s}_0} (\hat{\sigma}^2 \hat{V})^{-1}(\symbf{Z}-\hat{\mu}\symbf{1}),
\label{eq:prediction}
\end{equation} where
\(\symbf{C}_{\symbf{s}_0}=\hat{\sigma}^2(\exp(-\|\symbf{s}_1-\symbf{s}_0\|/\hat{\eta}),\dots,\exp(-\|\symbf{s}_n-\symbf{s}_0\|/\hat{\eta}))\prime\).
The corresponding prediction variance is
\(\mbox{Var}(\hat{S}(\symbf{s}_0)-S(\symbf{s}_0))=\hat{\sigma}^2-\symbf{C}\prime_{\symbf{s}_0} (\hat{\sigma}^2 \hat{V})^{-1}\symbf{C}_{\symbf{s}_0}\),
based on which we have the inverse variance for the prediction,
\(\zeta(\symbf{s}_0)=\frac{1}{\mbox{Var}(\hat{S}(\symbf{s}_0)-S(\symbf{s}_0))}\).
The higher \(\zeta(\symbf{s}_0)\) is, the better quality the prediction
has, and we give more weight to the most reliable pollution values while
aggregating them \autocites{SanchezMeca1998}{LeeC2016}.

After obtaining pollution predictions at the center of all grids where
the population density data are available (see Figure \ref{fig:maps2})
using (\ref{eq:prediction}), and denoting the population density at
location \(\symbf{s}_i\) as \(G(\symbf{s}_i)\), for a specific county
\(k\), the spatially representative pollution concentration is estimated
by \begin{equation}
z_k = \sum_{\symbf{s}_i \in A_k}\frac{\hat{S}(\symbf{s}_i)G(\symbf{s}_i)\zeta(\symbf{s}_i)}{\sum_{\symbf{s}_j \in A_k} G(\symbf{s}_j)\zeta(\symbf{s}_i)},
\label{eq:arealEstimate}
\end{equation} where \(A_k\) represents county \(k\). Therefore, \(z_k\)
is a spatial metric of pollution concentrations weighted by population
density and also the inverse of their Kriged variances.

\hypertarget{sec:healthModel}{%
\subsection{Health model}\label{sec:healthModel}}

Recall that the disease data are counts of the numbers of cases
occurring in each county in Germany, and that the observed and expected
number of COVID-19 cases for county \(k\) are denoted as \(Y_{k}\) and
\(E_{k}\), respectively. The health model is a Poisson log-linear model
\autocite[see][]{Shaddick2015}, given by

\begin{eqnarray}\label{eq:healthmodel}
Y_{k}  &\sim & \mbox{Poisson}(E_{k}\lambda_{k}),~~k=1,\ldots,K,\\\nonumber
\log(\lambda_{k}) &=& \symbf{X}_{k}^{\top}\symbf{\beta}+\phi_{k}
\end{eqnarray} where the relative risk of disease in country \(k\) is
denoted by \(\lambda_{k}\), and is modelled on the log scale by
covariates \(\symbf{X}_{k}\), containing an intercept column and
covariates (pollutants, temperature and areal population density
{[}population divided by area{]} referred to as popDensity), and a
spatial random effect \(\phi_{k}\). The regression parameters
\(\symbf{\beta}\) are assigned weakly informative zero-mean Gaussian
priors with diagonal variance matrix
\(\symbf{\beta}\sim \mbox{N}\left(\symbf{0},10^2\symbf{I} \right)\).

The spatial random effect, \(\phi_{k}\), is included to allow for any
residual spatial autocorrelation remaining in the disease counts after
the covariate effects have been accounted for, and is modelled by,
\begin{eqnarray}\label{eq:model2}
\symbf{\phi} \sim \mbox{N}\left(\symbf{0},\kappa^2 \symbf{Q}(\xi,\symbf{W})^{-1}\right),
\end{eqnarray} where \(\symbf{\phi}=\{\phi_k, k=1,\dots, K\}\). Spatial
autocorrelation is induced into the random effects by the precision
matrix
\(\symbf{Q}(\xi,\symbf{W})=\xi(\mbox{diag}(\symbf{W}\symbf{1})-\symbf{W})+(1-\xi)\symbf{I}\),
which corresponds to the CAR model proposed by \textcite{Leroux1999}.
The spatial dependence in the data is captured by an \(K \times K\)
neighbourhood matrix \(\symbf{W}\), whose \(ij\)th element equals 1 if
areas \((i,j )\) share a common border and is zero otherwise. The level
of spatial autocorrelation in the random effects is controlled by
\(\xi\). Finally, weakly informative hyperpriors are specified for the
parameters \((\kappa^2,\xi)\) by \begin{eqnarray}\label{eq:model3}
\kappa &\sim &  \mbox{Exp}[\log(2)],\\\nonumber
\log\left(\frac{\xi}{1-\xi}\right) &\sim & \mbox{N}(0,1.8).
\end{eqnarray} The prior distribution of \(\xi\) is likely
non-informative as it is roughly uniformly distributed within {[}0,1{]}.
The prior distribution of \(\kappa\) allows small values, which are what
we expected for the variation of the log scale of relative risk. Health
models are implemented in INLA which uses the Integrated Nested Laplace
Approximation \autocite{Rue2009}, a computationally effective and
extremely powerful alternative to implement Bayesian models, that is an
increasingly popular analysis package in R. For details on how to fit
spatial and spatio-temporal models with R-INLA, refer to
\textcite{Blangiardo2013}.

\hypertarget{sec:results}{%
\section{Results}\label{sec:results}}

\hypertarget{exposure-estimation}{%
\subsection{Exposure estimation}\label{exposure-estimation}}

Table \ref{tab:polParameter} presents the estimation of pollution model
parameters while applying model (\ref{eq:model}) to different pollutants
separately. The main message from the table is that the Akaike
information criterion \autocite[AIC,][]{Akaike1973} from the proposed
spatial pollution model (\ref{eq:model}) are all well less than those
from non-spatial models (an intercept with independent errors, without
spatial component \(R\) in model (\ref{eq:var})), indicating the
necessity of the spatial structure in pollution model. The main results
from pollution model are the population-weighted county level exposure
for each pollutant, with an example of NO\(_2\) population-weighted
exposure being shown in Figure \ref{fig:maps4} that the east part of
Germany has much higher NO\(_2\) exposure levels. A summary of the
estimated population-weighted county level exposure is presented in
Table \ref{tab:dataTable}, and an exploratory of the scatterplots of the
natural logarithm of COVID-19 SIR against the population-weighted
NO\(_2\) and PM\(_{2.5}\) are displayed in the upper of Figure
\ref{fig:modelledposterior}, which appears to indicate a linear
relationship between NO\(_2\) and log COVID-19 SIR.

\begin{table}

\caption{\label{tab:polParameter}Parameter estimation from pollution model (\ref{eq:model}).}
\centering
\begin{tabular}[t]{lrrrrrr}
\toprule
\textbf{ } & \textbf{$\symbf{\mu}$} & \textbf{$\symbf{\sigma^2}$} & \textbf{$\symbf{\tau^2}$} & \textbf{$\symbf{\eta}$} & \textbf{AIC} & \textbf{Non-spatial AIC}\\
\midrule
NO$_2$ & 20.181 & 60.532 & 107.134 & 0.537 & 4501.620 & 4644.213\\
PM$_{2.5}$ & 10.395 & 1.156 & 1.525 & 0.736 & 741.625 & 771.677\\
PM$_{10}$ & 17.483 & 4.375 & 10.265 & 0.554 & 2138.916 & 2178.230\\
SO$_2$ & 1.507 & 0.751 & 0.115 & 0.681 & 301.668 & 365.301\\
Benzene & 0.846 & 0.022 & 0.093 & 1.584 & 96.896 & 107.443\\
Aresenic & 0.446 & 0.014 & 0.031 & 1.473 & -71.610 & -54.280\\
Cadmium & 0.110 & 0.001 & 0.002 & 1.159 & -532.236 & -499.639\\
Nickel & 1.468 & 0.532 & 1.239 & 0.922 & 603.834 & 631.438\\
Temperature & 9.810 & 1.527 & 0.245 & 0.857 & 1783.719 & 2175.323\\
\bottomrule
\end{tabular}
\end{table}

\begin{table}

\caption{\label{tab:dataTable}Population-weighted county level exposure summary, with unit $\mu g m^{-3}$ for NO$_2$, PM$_{25}$, PM$_{10}$, SO$_2$, Benzene; $ng m^{-3}$ for Aresenic, admium, Nickel; and $^{\circ}C$  for temperature.}
\centering
\begin{tabular}[t]{lrrrrrr}
\toprule
 & \textbf{Min} & \textbf{Quantile25} & \textbf{Median} & \textbf{Mean} & \textbf{Quantile75} & \textbf{Max}\\
\midrule
NO$_2$ & 12.57 & 18.79 & 21.62 & 23.03 & 26.86 & 36.54\\
PM$_{2.5}$ & 8.64 & 9.98 & 10.52 & 10.48 & 10.94 & 12.21\\
PM$_{10}$ & 14.48 & 16.84 & 17.74 & 17.74 & 18.57 & 21.07\\
SO$_2$ & 0.69 & 1.21 & 1.51 & 1.66 & 1.95 & 4.23\\
Benzene & 0.69 & 0.81 & 0.85 & 0.88 & 0.96 & 1.15\\
Aresenic & 0.32 & 0.40 & 0.44 & 0.45 & 0.50 & 0.67\\
Cadmium & 0.08 & 0.10 & 0.10 & 0.11 & 0.13 & 0.20\\
Nickel & 0.97 & 1.32 & 1.44 & 1.63 & 1.85 & 3.22\\
Temperature & 6.69 & 9.61 & 10.15 & 10.09 & 10.54 & 11.84\\
\bottomrule
\end{tabular}
\end{table}

\hypertarget{health-model-validation}{%
\subsection{Health model validation}\label{health-model-validation}}

Before presenting the health effects results from health model
(\ref{eq:healthmodel}), we assess the necessity of including spatial
autocorrelation via the random effects model (\ref{eq:model2}) by
fitting a simplified version of model (\ref{eq:healthmodel}) without
spatial random effects term \(\phi_k\). The residuals from this model
show substantial spatial autocorrelation, with significant Moran's I
statistics (see the middle of Figure \ref{fig:modelledposterior})
\autocite{Moran1950}. The empirical semi-variogram of the residuals in
Figure \ref{fig:modelledposterior4} shows that few points are lying
outside the 95\% Monte Carlo simulation envelopes, suggesting strong
spatial autocorrelation is remained in the residuals and including the
spatial random effect term (\ref{eq:model2}) in health model is
appropriate.

\hypertarget{pollution-health-effects}{%
\subsection{Pollution health effects}\label{pollution-health-effects}}

In this section, we present the air pollution health effects, which are
the main results in this study. For comparison purposes, we show both
the results from our employed health model with the
\textcite{Leroux1999} CAR model to account for spatially correlated
residuals (referred to as ``Leroux''), and the results from other
commonly used CAR models, including the intrinsic autoregressive
proposed by \textcite{Besag1991} (referred to as ``Besag''), convolution
model also proposed by \textcite{Besag1991} (referred to as ``BYM''),
and also the non-spatial model (referred to as ``IID''). In addition, as
PM\(_{10}\) and PM\(_{2.5}\) are highly correlated (with correlation
coefficient being 0.75 in our study), we run two separated health model
to avoid collinearity, with each model including either PM\(_{10}\) or
PM\(_{2.5}\) and all the other covariates. The results from having
PM\(_{2.5}\) in the model are presented in this section in Table
\ref{tab:monitoringModel}, while those from having PM\(_{10}\) are
presented in the Appendix in Table \ref{tab:monitoringModelpm10}.

The bottom of Figure \ref{fig:modelledposterior} shows both the prior
and posterior distributions of the spatial dependence parameter \(\xi\)
and variance parameter \(\kappa\) from fitting the health model
(\ref{eq:healthmodel}) (having PM\(_{2.5}\)), respectively, suggesting
that both of them are well estimated from the data. Figure
\ref{fig:modelledposterior5} shows that the estimate of \(\xi\) is
around 0.8 which indicates high spatial autocorrelation in the disease
data after the covariate effects have been accounted for, validating the
use of the spatial random effects model (\ref{eq:model2}). Similarly,
Figure \ref{fig:modelledposterior6} shows the estimate of the spatial
variance parameter \(\kappa\) is around 0.75. The predicted COVID-19 SIR
presented in Figure \ref{fig:modelledSIR1} from health model
(\ref{eq:healthmodel}) shows that the majority of the high‐risk counties
are at the east and south part of Germany, that is where the counties
likely to have high probabilities of excess risk (see Figure
\ref{fig:modelledSIR2}).

The main results are presented in Table \ref{tab:monitoringModel},
including the posterior medians and 95\% credible intervals of relative
risk from one-unit increase for each covariate, and the widely
applicable information criterion (WAIC) \autocite{Watanabe2010} from
fitting various health models, including the employed Leroux model, and
commonly used BYM, Besag, IID models. Table \ref{tab:monitoringModel}
shows that the WAIC from different CAR models are similar, while it is
slightly lower (better) from the currently used Leroux model. The
results from Leroux model show that NO\(_2\) is significantly (at 0.05
level) associated with the COVID-19 SIR, with 1 \(\mu gm^{-3}\) increase
of long-term exposure to NO\(_2\), the COVID-19 incidence rate is likely
to increase 5.58\% (95\% CI: 3.35\%, 7.86\%). This statistically
significant association between NO\(_2\) and COVID-19 SIR is consistent
across various health models, including the BYM, Besag, IID models (and
also those from Table \ref{tab:monitoringModelpm10} where health model
has PM\(_{10}\) rather than PM\(_{2.5}\)), which enhances the
plausibility of the results.

The areal population density dose not have a significant association
with COVID-19 SIR, while temperature displays a negative association (at
0.05 level) with COVID-19 incidence rate. As shown in Figure
\ref{fig:maps3}, the COVID-19 SIRs are generally higher in the south
where actually has a lower long-term temperature surface compared to the
north \autocite[see][]{mapsofworld2020}. This explains the negative
association between temperature and COVID-19 incidence rate. No
substantial associations (at 0.05 level) were found between COVID-19
incidence rate and the other pollutants, including PM\(_{2.5}\),
SO\(_2\), Benzene, Aresenic, Cadmium and Nickel. Note that SO\(_2\) is
just at the border of having a significant association with COVID-19
SIR, since the posterior probabilities of its increasing relative risk
is 0.96 (see Table \ref{tab:monitoringModel}). And SO\(_2\) is
significantly associated with COVID-19 SIR from the model having
PM\(_{10}\) rather than PM\(_{2.5}\) (see the Leroux model results from
Table \ref{tab:monitoringModelpm10}).

\newgeometry{left=3.5cm}

\begin{landscape}\begin{table}

\caption{\label{tab:monitoringModel}Posterior medians and 95\% CI for the relative risk (\%) from one-unit increase in each covariate, and the WAIC from fitting various health models (having PM$_{2.5}$), including the employed Leroux model, and commonly used BYM, Besag, IID models. Pr is the posterior probabilities that covariate increases relative risk.}
\centering
\fontsize{9}{11}\selectfont
\begin{tabular}[t]{lrlcrlcrlcrlc}
\toprule
\multicolumn{1}{c}{\textbf{ }} & \multicolumn{3}{c}{\textbf{Leroux}} & \multicolumn{3}{c}{\textbf{BYM}} & \multicolumn{3}{c}{\textbf{Besag}} & \multicolumn{3}{c}{\textbf{IID}} \\
\cmidrule(l{3pt}r{3pt}){2-4} \cmidrule(l{3pt}r{3pt}){5-7} \cmidrule(l{3pt}r{3pt}){8-10} \cmidrule(l{3pt}r{3pt}){11-13}
\multicolumn{1}{c}{ } & \multicolumn{1}{c}{Est} & \multicolumn{1}{c}{CI} & \multicolumn{1}{c}{Pr} & \multicolumn{1}{c}{Est} & \multicolumn{1}{c}{CI} & \multicolumn{1}{c}{Pr} & \multicolumn{1}{c}{Est} & \multicolumn{1}{c}{CI} & \multicolumn{1}{c}{Pr} & \multicolumn{1}{c}{Est} & \multicolumn{1}{c}{CI} & \multicolumn{1}{c}{Pr} \\
\cmidrule(l{3pt}r{3pt}){2-2} \cmidrule(l{3pt}r{3pt}){3-3} \cmidrule(l{3pt}r{3pt}){4-4} \cmidrule(l{3pt}r{3pt}){5-5} \cmidrule(l{3pt}r{3pt}){6-6} \cmidrule(l{3pt}r{3pt}){7-7} \cmidrule(l{3pt}r{3pt}){8-8} \cmidrule(l{3pt}r{3pt}){9-9} \cmidrule(l{3pt}r{3pt}){10-10} \cmidrule(l{3pt}r{3pt}){11-11} \cmidrule(l{3pt}r{3pt}){12-12} \cmidrule(l{3pt}r{3pt}){13-13}
NO$_2$ & 5.58 & (  3.35,   7.86) & [1.00] & 5.21 & (  2.98,   7.50) & [1.00] & 5.36 & (  3.06,   7.71) & [1.00] & 5.60 & (  3.94,   7.29) & [1.00]\\
PM$_{2.5}$ & 4.59 & (-12.57,  24.79) & [0.69] & 1.62 & (-15.76,  22.51) & [0.57] & 0.45 & (-17.49,  22.27) & [0.52] & 8.04 & ( -2.70,  19.94) & [0.93]\\
SO$_2$ & 15.83 & ( -1.42,  35.45) & [0.96] & 6.29 & ( -9.20,  24.36) & [0.78] & 5.45 & (-10.56,  24.31) & [0.74] & 39.51 & ( 26.44,  53.94) & [1.00]\\
Temperature & -11.72 & (-20.84,  -1.46) & [0.01] & -8.52 & (-18.01,   2.05) & [0.05] & -8.48 & (-18.23,   2.41) & [0.06] & -18.12 & (-24.55, -11.16) & [0.00]\\
Benzene & -1.21 & (-19.21,  20.12) & [0.45] & -2.75 & (-23.00,  22.75) & [0.41] & -3.45 & (-24.59,  23.58) & [0.39] & 9.32 & ( -1.58,  21.41) & [0.95]\\
Aresenic & -16.72 & (-31.45,   1.67) & [0.04] & -9.34 & (-26.32,  11.47) & [0.17] & -10.27 & (-27.89,  11.64) & [0.16] & -22.38 & (-30.83, -12.92) & [0.00]\\
Cadmium & 16.44 & ( -5.67,  44.53) & [0.92] & 23.93 & ( -1.12,  55.49) & [0.97] & 27.08 & (  0.21,  61.15) & [0.98] & 6.86 & ( -5.49,  20.80) & [0.86]\\
Nickel & -1.35 & (-13.13,  12.03) & [0.42] & -1.41 & (-13.44,  12.26) & [0.41] & -1.56 & (-14.22,  12.95) & [0.41] & -1.47 & ( -8.90,   6.57) & [0.35]\\
popDensity & -2.12 & ( -7.34,   3.39) & [0.22] & -2.23 & ( -7.37,   3.19) & [0.20] & -1.83 & ( -6.98,   3.61) & [0.25] & -6.15 & (-11.35,  -0.65) & [0.01]\\
WAIC &  & 3814.64 &  &  & 3814.8 &  &  & 3816.18 &  &  & 3815.05 & \\
\bottomrule
\end{tabular}
\end{table}
\end{landscape}
\restoregeometry

\hypertarget{sec:conclusion}{%
\section{Discussion}\label{sec:conclusion}}

``Poisoning our environment means poisoning our own body, and when it
experiences chronic respiratory stress its ability to defend itself from
infections is limited'' \autocite{Ogen2020}. Given that the existing
research has linked pollutants (e.g., PM\(_{2.5}\) and NO\(_2\))
exposure to health damage, particularly respiratory and lung diseases,
which could make people more vulnerable to contracting COVID-19. This
study uses a spatial ecological design to estimate the impacts of air
pollution on COVID-19 infection in Germany by comparing geographical
contrasts in air pollution and infection risk across \(K\) contiguous
small areas, where we use population-weighted method to better estimate
the real areal pollution concentrations. The results show that long-term
exposure to NO\(_2\) is significantly associated with COVID-19 incidence
rate in Germany, with 1 \(\mu gm^{-3}\) increase of long-term exposure
to NO\(_2\), the COVID-19 incidence rate is likely to increase 5.58\%
(95\% CI: 3.35\%, 7.86\%). No substantial associations were found
between COVID-19 incidence rate and the other pollutants, including
PM\(_{2.5}\), PM\(_{10}\), SO\(_2\), Benzene, Aresenic, Cadmium and
Nickel. Temperature and population density are adjusted in the model,
and a set of random effects are also included to capture the residual
spatial autocorrelation after the covariate effects have been accounted
for.

For comparison purposes, we also run the health models with other
commonly used CAR models, including the intrinsic autoregressive
proposed by \textcite{Besag1991}, convolution model also proposed by
\textcite{Besag1991}, and also the non-spatial IID model. In addition,
as PM\(_{2.5}\) and PM\(_{10}\) are highly correlated, we run two
separated health models to avoid collinearity, with each model including
either PM\(_{2.5}\) or PM\(_{10}\) and all the other covariates. We
found that the statistically significant associations between NO\(_2\)
and COVID-19 SIR are consistent across these various health models,
which enhances the plausibility of the results.

Several limitations to this pilot study need to be acknowledged. First,
due to data availability, no socioeconomic or health care related
covariates were included in the health model which, if included, would
provide the possibility of sensitivity analyses and help testing the
robustness of the findings. However, in our health model, we do include
a spatial random effects term to allow for any spatial autocorrelation
residuals after accounting for the known covariates, and the main
findings of NO\(_2\) are actually adjusted for a set of other pollutants
and temperature. Another limitation is lacking COVID-19 testing numbers,
since the confirmed cases (positive testing numbers) in a county mainly
rely on the total testing numbers being conducted in that county,
without which the infection rates in some counties could be higher or
lower estimated compared to others'. Therefore, we put an effort to
better estimate the expected number of cases in each county by utilizing
national sex-age standardized infection rate. The limitation in regard
to pollution model is that the current one is a univariate model, which
is potentially losing some power by not borrowing strength over
correlated pollutants compared to a multivariate pollution model.

Finally, besides COVID-19 infection rate, its death rate and
multi-country studies should also be focused when (or if) more deaths
occur in the future. Such studies will help us better understanding
COVID-19, and also help the global community and health organizations
stay informed and make data driven decisions.

\begin{figure}[H]
\subfloat[NO$_2$\label{fig:modelledposterior1}]{\includegraphics[width=0.5\textwidth]{figure/modelledposterior-1} }\subfloat[PM$_{2.5}$\label{fig:modelledposterior2}]{\includegraphics[width=0.5\textwidth]{figure/modelledposterior-2} }\subfloat[Moran's I test, p-value<0.0001\label{fig:modelledposterior3}]{\includegraphics[width=0.5\textwidth]{figure/modelledposterior-3} }\subfloat[Empirical semi-variogram\label{fig:modelledposterior4}]{\includegraphics[width=0.5\textwidth]{figure/modelledposterior-4} }\subfloat[$\xi$\label{fig:modelledposterior5}]{\includegraphics[width=0.5\textwidth]{figure/modelledposterior-5} }\subfloat[$\kappa$\label{fig:modelledposterior6}]{\includegraphics[width=0.5\textwidth]{figure/modelledposterior-6} }\caption{Upper: scatterplots of log COVID-19 SIR against NO$_2$ ($\mu gm^{-3}$) and PM$_{2.5}$ ($\mu gm^{-3}$); Middle: the Moran's I test and the empirical semi-variogram of the residuals from the non-spatial health model (circles), with 95\% Monte Carlo simulation envelopes (dashed lines); Bottom: posterior (solid line) and prior (dashed line) plots for $\xi$ and $\kappa$ from health model (\ref{eq:healthmodel}).}\label{fig:modelledposterior}
\end{figure}

\begin{figure}[H]
\subfloat[Predicted risk \label{fig:modelledSIR1}]{\includegraphics[width=0.5\textwidth]{figure/modelledSIR-1} }\subfloat[Exceedance probabilities\label{fig:modelledSIR2}]{\includegraphics[width=0.5\textwidth]{figure/modelledSIR-2} }\caption[Posterior means of relative risk E$(\lambda_k|Y)$ and probabilities of excess risk Pr$(\exp(\phi_i)> 1.5 \mid Y)$]{Posterior means of relative risk E$(\lambda_k|Y)$ and probabilities of excess risk Pr$(\exp(\phi_i)> 1.5 \mid Y)$.}\label{fig:modelledSIR}
\end{figure}

\hypertarget{appendix}{%
\section*{Appendix}\label{appendix}}
\addcontentsline{toc}{section}{Appendix}

The results from fitting health models having PM\(_{10}\) are shown in
Table \ref{tab:monitoringModelpm10}. The raw data and R code used in
this study will be shared on Github shortly.

\newgeometry{left=3.5cm}

\begin{landscape}\begin{table}

\caption{\label{tab:monitoringModelpm10}Posterior medians and 95\% CI for the relative risk (\%) from one-unit increase in each covariate, and the WAIC from fitting various health models (having PM$_{2.5}$), including the employed Leroux model, and commonly used BYM, Besag, IID models. Pr is the posterior probabilities that covariate increases relative risk.}
\centering
\fontsize{9}{11}\selectfont
\begin{tabular}[t]{lrlcrlcrlcrlc}
\toprule
\multicolumn{1}{c}{\textbf{ }} & \multicolumn{3}{c}{\textbf{Leroux}} & \multicolumn{3}{c}{\textbf{BYM}} & \multicolumn{3}{c}{\textbf{Besag}} & \multicolumn{3}{c}{\textbf{IID}} \\
\cmidrule(l{3pt}r{3pt}){2-4} \cmidrule(l{3pt}r{3pt}){5-7} \cmidrule(l{3pt}r{3pt}){8-10} \cmidrule(l{3pt}r{3pt}){11-13}
\multicolumn{1}{c}{ } & \multicolumn{1}{c}{Est} & \multicolumn{1}{c}{CI} & \multicolumn{1}{c}{Pr} & \multicolumn{1}{c}{Est} & \multicolumn{1}{c}{CI} & \multicolumn{1}{c}{Pr} & \multicolumn{1}{c}{Est} & \multicolumn{1}{c}{CI} & \multicolumn{1}{c}{Pr} & \multicolumn{1}{c}{Est} & \multicolumn{1}{c}{CI} & \multicolumn{1}{c}{Pr} \\
\cmidrule(l{3pt}r{3pt}){2-2} \cmidrule(l{3pt}r{3pt}){3-3} \cmidrule(l{3pt}r{3pt}){4-4} \cmidrule(l{3pt}r{3pt}){5-5} \cmidrule(l{3pt}r{3pt}){6-6} \cmidrule(l{3pt}r{3pt}){7-7} \cmidrule(l{3pt}r{3pt}){8-8} \cmidrule(l{3pt}r{3pt}){9-9} \cmidrule(l{3pt}r{3pt}){10-10} \cmidrule(l{3pt}r{3pt}){11-11} \cmidrule(l{3pt}r{3pt}){12-12} \cmidrule(l{3pt}r{3pt}){13-13}
NO$_2$ & 6.06 & (  3.50,   8.67) & [1.00] & 5.51 & (  2.91,   8.20) & [1.00] & 5.57 & (  2.94,   8.26) & [1.00] & 6.93 & (  5.02,   8.88) & [1.00]\\
PM$_{10}$ & -2.98 & (-12.49,   7.63) & [0.28] & -1.43 & (-11.56,   9.81) & [0.40] & -1.61 & (-11.78,   9.72) & [0.38] & -7.88 & (-14.13,  -1.17) & [0.01]\\
SO$_2$ & 18.82 & (  0.90,  39.05) & [0.98] & 6.40 & ( -9.64,  25.21) & [0.77] & 6.18 & ( -9.94,  25.18) & [0.76] & 50.66 & ( 37.18,  65.45) & [1.00]\\
Temperature & -10.89 & (-20.34,  -0.24) & [0.02] & -8.09 & (-18.09,   3.13) & [0.07] & -8.05 & (-18.12,   3.24) & [0.08] & -16.21 & (-22.97,  -8.87) & [0.00]\\
Benzene & -0.80 & (-18.69,  20.39) & [0.47] & -3.00 & (-23.67,  23.12) & [0.40] & -3.27 & (-24.07,  23.22) & [0.39] & 8.46 & ( -2.34,  20.43) & [0.94]\\
Aresenic & -13.74 & (-29.20,   5.35) & [0.07] & -9.18 & (-26.99,  12.85) & [0.19] & -9.42 & (-27.34,  12.88) & [0.19] & -12.82 & (-22.89,  -1.44) & [0.01]\\
Cadmium & 13.52 & ( -8.03,  41.21) & [0.88] & 25.46 & ( -0.87,  59.05) & [0.97] & 26.31 & ( -0.47,  60.27) & [0.97] & -1.37 & (-12.55,  11.21) & [0.41]\\
Nickel & -0.70 & (-12.51,  12.67) & [0.46] & -1.32 & (-13.82,  12.95) & [0.42] & -1.37 & (-14.00,  13.11) & [0.42] & -1.17 & ( -8.59,   6.85) & [0.38]\\
popDensity & -2.08 & ( -7.30,   3.43) & [0.22] & -1.92 & ( -7.09,   3.52) & [0.24] & -1.82 & ( -6.98,   3.61) & [0.25] & -5.89 & (-11.08,  -0.41) & [0.02]\\
WAIC &  & 3814.55 &  &  & 3815.65 &  &  & 3816.13 &  &  & 3814.58 & \\
\bottomrule
\end{tabular}
\end{table}
\end{landscape}
\restoregeometry

\printbibliography[title=References]

\end{document}
